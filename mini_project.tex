% Options for packages loaded elsewhere
\PassOptionsToPackage{unicode}{hyperref}
\PassOptionsToPackage{hyphens}{url}
%
\documentclass[
]{article}
\usepackage{amsmath,amssymb}
\usepackage{iftex}
\ifPDFTeX
  \usepackage[T1]{fontenc}
  \usepackage[utf8]{inputenc}
  \usepackage{textcomp} % provide euro and other symbols
\else % if luatex or xetex
  \usepackage{unicode-math} % this also loads fontspec
  \defaultfontfeatures{Scale=MatchLowercase}
  \defaultfontfeatures[\rmfamily]{Ligatures=TeX,Scale=1}
\fi
\usepackage{lmodern}
\ifPDFTeX\else
  % xetex/luatex font selection
\fi
% Use upquote if available, for straight quotes in verbatim environments
\IfFileExists{upquote.sty}{\usepackage{upquote}}{}
\IfFileExists{microtype.sty}{% use microtype if available
  \usepackage[]{microtype}
  \UseMicrotypeSet[protrusion]{basicmath} % disable protrusion for tt fonts
}{}
\makeatletter
\@ifundefined{KOMAClassName}{% if non-KOMA class
  \IfFileExists{parskip.sty}{%
    \usepackage{parskip}
  }{% else
    \setlength{\parindent}{0pt}
    \setlength{\parskip}{6pt plus 2pt minus 1pt}}
}{% if KOMA class
  \KOMAoptions{parskip=half}}
\makeatother
\usepackage{xcolor}
\usepackage[margin=1in]{geometry}
\usepackage{color}
\usepackage{fancyvrb}
\newcommand{\VerbBar}{|}
\newcommand{\VERB}{\Verb[commandchars=\\\{\}]}
\DefineVerbatimEnvironment{Highlighting}{Verbatim}{commandchars=\\\{\}}
% Add ',fontsize=\small' for more characters per line
\usepackage{framed}
\definecolor{shadecolor}{RGB}{248,248,248}
\newenvironment{Shaded}{\begin{snugshade}}{\end{snugshade}}
\newcommand{\AlertTok}[1]{\textcolor[rgb]{0.94,0.16,0.16}{#1}}
\newcommand{\AnnotationTok}[1]{\textcolor[rgb]{0.56,0.35,0.01}{\textbf{\textit{#1}}}}
\newcommand{\AttributeTok}[1]{\textcolor[rgb]{0.13,0.29,0.53}{#1}}
\newcommand{\BaseNTok}[1]{\textcolor[rgb]{0.00,0.00,0.81}{#1}}
\newcommand{\BuiltInTok}[1]{#1}
\newcommand{\CharTok}[1]{\textcolor[rgb]{0.31,0.60,0.02}{#1}}
\newcommand{\CommentTok}[1]{\textcolor[rgb]{0.56,0.35,0.01}{\textit{#1}}}
\newcommand{\CommentVarTok}[1]{\textcolor[rgb]{0.56,0.35,0.01}{\textbf{\textit{#1}}}}
\newcommand{\ConstantTok}[1]{\textcolor[rgb]{0.56,0.35,0.01}{#1}}
\newcommand{\ControlFlowTok}[1]{\textcolor[rgb]{0.13,0.29,0.53}{\textbf{#1}}}
\newcommand{\DataTypeTok}[1]{\textcolor[rgb]{0.13,0.29,0.53}{#1}}
\newcommand{\DecValTok}[1]{\textcolor[rgb]{0.00,0.00,0.81}{#1}}
\newcommand{\DocumentationTok}[1]{\textcolor[rgb]{0.56,0.35,0.01}{\textbf{\textit{#1}}}}
\newcommand{\ErrorTok}[1]{\textcolor[rgb]{0.64,0.00,0.00}{\textbf{#1}}}
\newcommand{\ExtensionTok}[1]{#1}
\newcommand{\FloatTok}[1]{\textcolor[rgb]{0.00,0.00,0.81}{#1}}
\newcommand{\FunctionTok}[1]{\textcolor[rgb]{0.13,0.29,0.53}{\textbf{#1}}}
\newcommand{\ImportTok}[1]{#1}
\newcommand{\InformationTok}[1]{\textcolor[rgb]{0.56,0.35,0.01}{\textbf{\textit{#1}}}}
\newcommand{\KeywordTok}[1]{\textcolor[rgb]{0.13,0.29,0.53}{\textbf{#1}}}
\newcommand{\NormalTok}[1]{#1}
\newcommand{\OperatorTok}[1]{\textcolor[rgb]{0.81,0.36,0.00}{\textbf{#1}}}
\newcommand{\OtherTok}[1]{\textcolor[rgb]{0.56,0.35,0.01}{#1}}
\newcommand{\PreprocessorTok}[1]{\textcolor[rgb]{0.56,0.35,0.01}{\textit{#1}}}
\newcommand{\RegionMarkerTok}[1]{#1}
\newcommand{\SpecialCharTok}[1]{\textcolor[rgb]{0.81,0.36,0.00}{\textbf{#1}}}
\newcommand{\SpecialStringTok}[1]{\textcolor[rgb]{0.31,0.60,0.02}{#1}}
\newcommand{\StringTok}[1]{\textcolor[rgb]{0.31,0.60,0.02}{#1}}
\newcommand{\VariableTok}[1]{\textcolor[rgb]{0.00,0.00,0.00}{#1}}
\newcommand{\VerbatimStringTok}[1]{\textcolor[rgb]{0.31,0.60,0.02}{#1}}
\newcommand{\WarningTok}[1]{\textcolor[rgb]{0.56,0.35,0.01}{\textbf{\textit{#1}}}}
\usepackage{graphicx}
\makeatletter
\def\maxwidth{\ifdim\Gin@nat@width>\linewidth\linewidth\else\Gin@nat@width\fi}
\def\maxheight{\ifdim\Gin@nat@height>\textheight\textheight\else\Gin@nat@height\fi}
\makeatother
% Scale images if necessary, so that they will not overflow the page
% margins by default, and it is still possible to overwrite the defaults
% using explicit options in \includegraphics[width, height, ...]{}
\setkeys{Gin}{width=\maxwidth,height=\maxheight,keepaspectratio}
% Set default figure placement to htbp
\makeatletter
\def\fps@figure{htbp}
\makeatother
\setlength{\emergencystretch}{3em} % prevent overfull lines
\providecommand{\tightlist}{%
  \setlength{\itemsep}{0pt}\setlength{\parskip}{0pt}}
\setcounter{secnumdepth}{-\maxdimen} % remove section numbering
\ifLuaTeX
  \usepackage{selnolig}  % disable illegal ligatures
\fi
\IfFileExists{bookmark.sty}{\usepackage{bookmark}}{\usepackage{hyperref}}
\IfFileExists{xurl.sty}{\usepackage{xurl}}{} % add URL line breaks if available
\urlstyle{same}
\hypersetup{
  pdftitle={Mini-Projet Etude et application de quelques schémas aux différences finies pour deux lois de conservation},
  pdfauthor={AMECK GUY-MAX DESIRE DOSSEH \& RIM ELMGHARI},
  hidelinks,
  pdfcreator={LaTeX via pandoc}}

\title{Mini-Projet Etude et application de quelques schémas aux
différences finies pour deux lois de conservation}
\author{AMECK GUY-MAX DESIRE DOSSEH \& RIM ELMGHARI}
\date{2024-02-04}

\begin{document}
\maketitle

On souhaite étudier, appliquer et voir le comportement de quelques
schémas aux différences finies pour deux équations relevant de lois de
conservation 1D définies sur un domaine \(\Omega = [0, L]\).

\hypertarget{equation-de-transport}{%
\section{1. Equation de transport}\label{equation-de-transport}}

On considère l'équation de transport soumise à des conditions aux
limites periodiques:

\[
(E_{1})\left\{
\begin{array}{ll}
\frac{\partial u}{\partial t} + a\frac{\partial u}{\partial x} = 0, & \forall x \in ]0,L[\  ; \ \forall t>0 \\
u(x, t=0) = u_0(x), & \forall x \in [0,L] \\
u(0, t) = u(L, t)\ ; \ \frac{\partial u}{\partial x}(L, t)=0 & \forall t > 0
\end{array}
\right.
\] 1) A l'aide de la methode des caracteristiques, determiner la
solution exacte \(u(x, t)\) du probleme \((E_{1})\).

Nous allons chercher une courbe caractéristique
\(\Gamma ((t(s), x(s))\), s étant le paramètre qui décrit la courbe, le
long de laquelle l'EDP devient un système d'EDO.

\[
\begin{aligned}
du &= \frac{\partial u}{\partial t}dt + \frac{\partial u}{\partial x}dx\\
\frac{du}{ds} &= \frac{\partial u}{\partial t}\frac{dt}{ds} + \frac{\partial u}{\partial x}\frac{dx}{ds}\\
\frac{du}{ds} &= -a\frac{\partial u}{\partial x}\frac{dt}{ds} + \frac{\partial u}{\partial x}\frac{dx}{ds}\\
\frac{du}{ds} &= \frac{\partial u}{\partial x}(\frac{dx}{ds}-a\frac{dt}{ds})
\end{aligned}
\]

On voit que si on impose \(\frac{dx}{ds}-a\frac{dt}{ds} = 0\), on a
\(\frac{du}{ds} = 0\), c'est à dire que u est constant le long de la
courbe caractéristique.

On a donc le système d'EDO suivant a resoudre:

\[
\left\{
\begin{array}{ll}
\frac{dx}{dt} = a \ \text{qui donne la courbe caractéristique }\Gamma\\
du = 0 \ \text{qui donne la solution u(x, t) sur cette courbe caractéristique} 
\end{array}
\right.
\] Courbes caractéristiques:

\[
\frac{dx}{dt} = a \ \text{donne} \ x(t) = at + \xi (avec \ \xi \ \text{une constante reelle d'integration})
\]

\begin{Shaded}
\begin{Highlighting}[]
\ImportTok{import}\NormalTok{ numpy }\ImportTok{as}\NormalTok{ np}
\ImportTok{import}\NormalTok{ matplotlib.pyplot }\ImportTok{as}\NormalTok{ plt}

\CommentTok{\# Define the values of a and xi}
\NormalTok{a }\OperatorTok{=} \DecValTok{2}
\NormalTok{xi\_values }\OperatorTok{=}\NormalTok{ [}\OperatorTok{{-}}\DecValTok{4}\NormalTok{, }\OperatorTok{{-}}\DecValTok{3}\NormalTok{, }\OperatorTok{{-}}\DecValTok{2}\NormalTok{, }\OperatorTok{{-}}\DecValTok{1}\NormalTok{, }\DecValTok{0}\NormalTok{, }\DecValTok{1}\NormalTok{, }\DecValTok{2}\NormalTok{, }\DecValTok{3}\NormalTok{, }\DecValTok{4}\NormalTok{]}

\CommentTok{\# Define the time range}
\NormalTok{t }\OperatorTok{=}\NormalTok{ np.linspace(}\OperatorTok{{-}}\DecValTok{10}\NormalTok{, }\DecValTok{10}\NormalTok{, }\DecValTok{100}\NormalTok{)}

\CommentTok{\# Plot the characteristic curves for each xi value}
\ControlFlowTok{for}\NormalTok{ xi }\KeywordTok{in}\NormalTok{ xi\_values:}
\NormalTok{    x }\OperatorTok{=}\NormalTok{ a }\OperatorTok{*}\NormalTok{ t }\OperatorTok{+}\NormalTok{ xi}
\NormalTok{    plt.plot(t, x, label}\OperatorTok{=}\SpecialStringTok{f"xi = }\SpecialCharTok{\{}\NormalTok{xi}\SpecialCharTok{\}}\SpecialStringTok{"}\NormalTok{)}

\CommentTok{\# Plot the x and y axes}
\NormalTok{plt.axhline(}\DecValTok{0}\NormalTok{, color}\OperatorTok{=}\StringTok{\textquotesingle{}black\textquotesingle{}}\NormalTok{)}
\NormalTok{plt.axvline(}\DecValTok{0}\NormalTok{, color}\OperatorTok{=}\StringTok{\textquotesingle{}black\textquotesingle{}}\NormalTok{)}

\NormalTok{plt.xlabel(}\StringTok{\textquotesingle{}Time\textquotesingle{}}\NormalTok{)}
\NormalTok{plt.ylabel(}\StringTok{\textquotesingle{}x(t)\textquotesingle{}}\NormalTok{)}
\NormalTok{plt.title(}\StringTok{\textquotesingle{}Characteristic Curves\textquotesingle{}}\NormalTok{)}
\NormalTok{plt.grid(}\VariableTok{True}\NormalTok{)}
\NormalTok{plt.legend()}

\CommentTok{\# Set the x and y axes limits to center at 0}
\NormalTok{plt.xlim(}\OperatorTok{{-}}\DecValTok{10}\NormalTok{, }\DecValTok{10}\NormalTok{)}
\end{Highlighting}
\end{Shaded}

\begin{verbatim}
## (-10.0, 10.0)
\end{verbatim}

\begin{Shaded}
\begin{Highlighting}[]
\NormalTok{plt.ylim(}\OperatorTok{{-}}\DecValTok{10}\NormalTok{, }\DecValTok{10}\NormalTok{)}
\end{Highlighting}
\end{Shaded}

\begin{verbatim}
## (-10.0, 10.0)
\end{verbatim}

\begin{Shaded}
\begin{Highlighting}[]
\CommentTok{\# Set the x and y axis ticks}
\NormalTok{plt.xticks(np.arange(}\OperatorTok{{-}}\DecValTok{10}\NormalTok{, }\DecValTok{11}\NormalTok{, }\DecValTok{2}\NormalTok{))}
\end{Highlighting}
\end{Shaded}

\begin{verbatim}
## ([<matplotlib.axis.XTick object at 0x0000023DD3277ED0>, <matplotlib.axis.XTick object at 0x0000023DD3275A50>, <matplotlib.axis.XTick object at 0x0000023DCFCEDF90>, <matplotlib.axis.XTick object at 0x0000023DD33030D0>, <matplotlib.axis.XTick object at 0x0000023DD33090D0>, <matplotlib.axis.XTick object at 0x0000023DD330A350>, <matplotlib.axis.XTick object at 0x0000023DD330C310>, <matplotlib.axis.XTick object at 0x0000023DD330E450>, <matplotlib.axis.XTick object at 0x0000023DD3310650>, <matplotlib.axis.XTick object at 0x0000023DD3312710>, <matplotlib.axis.XTick object at 0x0000023DD3308810>], [Text(-10, 0, '−10'), Text(-8, 0, '−8'), Text(-6, 0, '−6'), Text(-4, 0, '−4'), Text(-2, 0, '−2'), Text(0, 0, '0'), Text(2, 0, '2'), Text(4, 0, '4'), Text(6, 0, '6'), Text(8, 0, '8'), Text(10, 0, '10')])
\end{verbatim}

\begin{Shaded}
\begin{Highlighting}[]
\NormalTok{plt.yticks(np.arange(}\OperatorTok{{-}}\DecValTok{10}\NormalTok{, }\DecValTok{11}\NormalTok{, }\DecValTok{2}\NormalTok{))}
\end{Highlighting}
\end{Shaded}

\begin{verbatim}
## ([<matplotlib.axis.YTick object at 0x0000023DD327FC50>, <matplotlib.axis.YTick object at 0x0000023DD16FCE50>, <matplotlib.axis.YTick object at 0x0000023DD16A4E10>, <matplotlib.axis.YTick object at 0x0000023DD331B190>, <matplotlib.axis.YTick object at 0x0000023DD5C21310>, <matplotlib.axis.YTick object at 0x0000023DD3319C50>, <matplotlib.axis.YTick object at 0x0000023DD5C23CD0>, <matplotlib.axis.YTick object at 0x0000023DD5C29D10>, <matplotlib.axis.YTick object at 0x0000023DD5C2BD50>, <matplotlib.axis.YTick object at 0x0000023DD5C2DE90>, <matplotlib.axis.YTick object at 0x0000023DD5C2FFD0>], [Text(0, -10, '−10'), Text(0, -8, '−8'), Text(0, -6, '−6'), Text(0, -4, '−4'), Text(0, -2, '−2'), Text(0, 0, '0'), Text(0, 2, '2'), Text(0, 4, '4'), Text(0, 6, '6'), Text(0, 8, '8'), Text(0, 10, '10')])
\end{verbatim}

\begin{Shaded}
\begin{Highlighting}[]
\CommentTok{\# Hide the legend}
\NormalTok{plt.legend().set\_visible(}\VariableTok{False}\NormalTok{)}
\NormalTok{plt.show()}
\end{Highlighting}
\end{Shaded}

\includegraphics{mini_project_files/figure-latex/unnamed-chunk-1-1.pdf}

On discretise l'intervalle \([0, L]\) en \((N-1)\) sous-interalles
\([x_{i}, x_{i+1}](i=1, ..., N-1)\) de tailles egales
\(\Delta x(\Delta x=\frac{L}{N-1}, x_{i+1}=x_{i}+\Delta x)\), et on note
par \(u_{i}^{n}\) la solution approchee au noeud \(x_{i}\) a l'instant
\(t^{n}=n\Delta t\)(\(\Delta t\) etant le pas de chnage).

\begin{enumerate}
\def\labelenumi{\arabic{enumi})}
\setcounter{enumi}{1}
\tightlist
\item
  Etudier la consistance, la stabilite et la convergence de chacun des
  schemas numeriques suivants:
\end{enumerate}

Schema 1 (centre):

\[
\frac{u_{j}^{n+1}-u_{j}^{n}}{\Delta t} + a\frac{u_{j+1}^{n}-u_{j-1}^{n}}{2\Delta x} = 0
\]

Schema 2 (decentre):

\[
\begin{aligned}
\frac{u_{j}^{n+1}-u_{j}^{n}}{\Delta t} + a\frac{u_{j}^{n}-u_{j-1}^{n}}{\Delta x} = 0 & \text{ si } a > 0\\
\frac{u_{j}^{n+1}-u_{j}^{n}}{\Delta t} + a\frac{u_{j+1}^{n}-u_{j}^{n}}{\Delta x} = 0 & \text{ si } a < 0
\end{aligned}
\]

Schema 3 (Lax-Friedrichs):

\[
\frac{u_{j}^{n+1}-\frac{1}{2}(u_{j-1}^{n}+u_{j+1}^{n})}{\Delta t} + a\frac{u_{j+1}^{n}-u_{j-1}^{n}}{2\Delta x} = 0
\]

Schema 4 (Lax-Wendroff):

\[
\frac{u_{j}^{n+1}-u_{j}^{n}}{\Delta t} + a\frac{u_{j+1}^{n}-u_{j-1}^{n}}{2\Delta x} - \frac{a^{2}\Delta t}{2(\Delta x)^{2}}(u_{j-1}^{n}-2u_{j}^{n}+u_{j+1}^{n}) = 0
\]

\begin{enumerate}
\def\labelenumi{\arabic{enumi})}
\setcounter{enumi}{2}
\item
  Implémenter chacun des schémas numériques pour évaluer la solution
  approchée, puis comparer cette solution avec la solution exacte.
  (Tracer les solutions aux temps physiques t1 = 2.5 s et t2 = 4.5 s en
  testant sur deux maillages différents formés de N = 100 et N = 200
  points. Interpréter les résultats.
\item
  Evaluer l'erreur en norme \(L^1\) de la solution numérique obtenue par
  chaque schéma au temps \(t_1 = 2.5 s\) et pour N = 100. Interpréter.
\end{enumerate}

\textbf{Données} :

\(L = 10 m ,\  a = 2 m/s , \ u_0(x) = 1\  pour\  3 m \leq x \leq 4 m\  et\  0 \ ailleurs.\)

Nombre de Courant : \(CFL = 0.8\).

\hypertarget{equation-de-burgers}{%
\section{2. Equation de Burgers}\label{equation-de-burgers}}

On considere maintenant l'equation de Burgers suivante:

\[
(E_{2})\left\{
\begin{array}{ll}
\frac{\partial u}{\partial t} + u\frac{\partial u}{\partial x} = 0, & \forall x \in ]0,L[\  ; \ \forall t>0 \\
u(x, t=0) = u_0(x), & \forall x \in [0,L] \\
\frac{\partial u}{\partial x}(0, t)= \frac{\partial u}{\partial x}(L, t)=0 & \forall t > 0
\end{array}
\right.
\] 5) Reprendre les questions 1), 3) et 4).

\textbf{Données} :

\(L = 6m ,\  u_0(x) = 0.4\  pour\  x < 2m \ et \ 0.1\ ailleurs.\ CFL = 0.8\).

\end{document}
